\documentclass[a4paper,12pt,fleqn]{article}
\author{Alastair Horn}
\title{A N S W E R S}

\usepackage{mysty}
\usepackage{mymaths}

% Embed source files into PDF in case of loss. You can view or extract the
% source files by doing `pdfdetach -list <file.pdf>` or
% `pdfdetach -saveall <file.pdf>`, using pdfdetach from poppler, or some other
% suitable method.
\usepackage{embedall}
\embedfile{mymaths.sty}
\embedfile{mysty.sty}

\begin{document}
\maketitle

\begin{enumerate}
 \item
  \adjustbox{valign=t}{
   \begin{tcolorbox}
    Let \(f: \R \to \R\) be continuous. Supposing that
    \begin{equation*}
     \dv[2]<y>{x} = f(y),
    \end{equation*}
    find an equation for \(y(x)\) in terms of \(f\).
   \end{tcolorbox}
  }

  Multiplying by \(\fdv<y>{x}\), we obtain
  \begin{alignat*}2
   && \dv<y>{x} \dv[2]<y>{x} &{}= \dv<y>{x}f(y) \\
   &\implies{}& \frac 12 \dv{x} \bracks[\Big]{\parens[\Big]{\dv<y>{x}}^2}
    &{}= \dv{x} [F(y)] \tag{1} \label{eq_1}
  \end{alignat*}
  for \(F\) an antiderivative of \(f\) - eg
  \begin{equation*}
   F(t) \defeq \integ[0]<t>{f(\nu)}{\nu}.
  \end{equation*}
  Then we can integrate (\ref{eq_1}) to get
  \begin{alignat*}2
   && \parens[\Big]{\dv<y>{x}}^2 &{}= 2F(y) + C_1 \\
   &\implies{}& \dv<y>{x} &{}= \pm \sqrt{2F(y) + C_1} \\
   &\implies{}& \integ{(2F(y) + C_1)^{-1/2}}{y} &{}= \pm x + C_2.
  \end{alignat*}

 \item
  % This is an equidimensional equation. The method will work fine for the
  % most general form of the equidimensional equation, but it gets pretty scary
  % when auxilliary equations have complex roots, and there is some casework
  % so I just picked one that looked as easy as possible
  \adjustbox{valign=t}{
   \begin{tcolorbox}
    Solve the differential equation for \(y\) in terms of \(x\):
    \begin{equation*}
      x^2\dv[2]<y>{x} + x\dv<y>{x} - y = 0
    \end{equation*}
   \end{tcolorbox}
  }

  Use substitution \(\displaystyle u = y + x\dv<y>{x}\). This transforms the
  equation into: \(\displaystyle x\dv<u>{x} - u = 0\) which is separable with solution
  \(u = 2Ax\) for some constant \(A\). Observe that we also have a first integral
  for \(y\): \(\displaystyle u = 2Ax = \frac{\mathrm{d}}{\mathrm{d}x}(xy)\),
  so integrating, we get \(xy = Ax^2 + B\) for some constant \(B\).
  So we are left with solution \(y = Ax + Bx^{-1}\).

 \item
  \adjustbox{valign=t}{
   \begin{tcolorbox}
    Let \(f: \R \to \R\) be continuous. Supposing that
    \begin{equation*}
     \dv[2]<y>{x} = f(y),
    \end{equation*}
    find an equation for \(y(x)\) in terms of \(f\).
   \end{tcolorbox}
  }

  Letting \(x = y = 0\), we see that \(f(0)^2 = f(0)\). Then either
  \begin{itemize}
   \item
    \(f(0) = 0\). In this case, letting \(y = 0\) shows that
    \(f(x) = f(x + 0) + x \cdot 0 =  f(x)f(0) = 0\) for all \(x\). So \(f\) is
    identically zero. However, letting \(x = y = 1\), we see that
    \(0 \ne 0 + 1\), so this does not provide a solution.
   \item
    or, \(f(0) = 1\). In this case, letting \(y = -x\) shows that
    \(f(x)f(-x) = f(0) - x^2 = 1 - x^2\), for all \(x\).

    Particularly it follows that \(f(1) f(-1) = 0\). So either
    \begin{itemize}
     \item
      \(f(1) = 0\). In this case, letting \(y = 1\), we have
      \(f(x + 1) + x = f(x) f(1) = 0\) for all \(x\), so
      \(f(x + 1) = -x\) for all \(x\), ie \(f(x) = 1 - x\) for all \(x\). And
      indeed, \((1 - x)(1 - y) = (1 - (x + y)) + xy\).
     \item
      or, \(f(-1) = 0\). In this case, letting \(y = -1\), we have
      \(f(x - 1) - x = f(x) f(-1) = 0\) for all \(x\), so
      \(f(x - 1) = x\) for all \(x\), ie \(f(x) = 1 + x\) for all \(x\). And
      indeed, \((1 + x)(1 + y) = (1 + (x + y)) + xy\).
    \end{itemize}
  \end{itemize}
  So either \(f(x) = 1 + x\) or \(f(x) = 1 - x\).
 \item
  \adjustbox{valign=t}{
   \begin{tcolorbox}
     Find all functions \(f: \R \to \R\), which satisfy the equation
     \[f(x^3) + f(y^3) = (x + y)(f(x^2) + f(y^2) - f(xy))\] for all real numbers
     \(x\) and \(y\).
   \end{tcolorbox}
  }

  Letting \(x = y = 0\) we discover the value of \(f(0)\) has to be \(0\).
  Letting \(y = -x\) gives \(f(x^3) = -f(-x^3)\), and since \(x^3\) surjective
  we therefore have \(f\) is odd, so for all \(x\), \(f(-x) = -f(x)\).

  Letting \(y = x\) gives \(2f(x^3) = 2x(2f(x^2) - f(x^2))\), so we have
  \(f(x^3) = xf(x^2)\) for all \(x\). Substituting this expression in for
  \(f(x^3)\) and \(f(y^3)\) in the original equation gives
  \(xf(x^2) + yf(y^2) = (x + y)(f(x^2) + f(y^2) - f(xy)\) which after expanding
  and cancelling gives
  \begin{equation} \label{eq:1}
   yf(x^2) + xf(y^2) = (x + y)f(xy)
  \end{equation}
  for all \(x\) and \(y\).

  Now let \(y = 1\), in this equation and we have
  \(f(x^2) + xf(1) = (x+1)f(x)\) for all \(x\), which gives us an expression for
  \(f(x^2)\): \(f(x^2) = (x+1)f(x) - xf(1)\). We can substitute these into
  \ref{eq:1}, giving us:
  \begin{equation*}
   y(x+1)f(x) + x(y+1)f(y) - 2xyf(1) = (x+y)f(xy)
  \end{equation*}

  Now substituting \(y = -x\) and using the fact that \(f\) is odd, we get:
  \begin{equation*}
    -2xf(x) + 2x^2f(1) = 0
  \end{equation*}
  So we have for \(x \neq 0\), \(f(x) = f(1)x\), and at \(0\) we know \(f\)
  takes the value \(0\) anyway. And we can check to see that this works for any
  chosen value of \(f(1)\). So \(f(x) = kx\) for some \(k \in \R\).
\end{enumerate}

\end{document}
